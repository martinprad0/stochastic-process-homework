\documentclass[12pt]{exam}

\usepackage{amsfonts, geometry, amsmath, amssymb, hyperref}
\usepackage{tikz, circuitikz, quiver, wrapfig}
\usepackage{algpseudocode, algorithm}

\geometry{ a4paper, top=2cm, bottom=2cm, left=2cm, right=2cm }

\parindent=0pt

\pagestyle{head}
%\pointsinmargin
\addpoints

%%---------------------------------------------------------------
%% AMS-LaTeX Paper 
%%---------------------------------------------------------------

\newcommand{\angles}[1]{\ensuremath{\left\langle}#1\ensuremath{\right\rangle} }

\def\N{\ensuremath{\mathbb{N}}}
\def\Z{\ensuremath{\mathbb{Z}}}
\def\Q{\ensuremath{\mathbb{Q}}}
\def\R{\ensuremath{\mathbb{R}}}
\def\C{\ensuremath{\mathbb{C}}}
\def\S{\ensuremath{\mathbb{S}}}

\usepackage{lastpage}
\usepackage[none]{hyphenat}

%%--MATH---------------------------------------------------------

%%--OTHER ENVIRONMENTS--------------------------------------
\renewcommand{\labelenumi}{{\bf \theenumi.}}
\renewcommand{\labelenumii}{{\bf (\theenumii)}}
\renewcommand{\labelenumiii}{{\bf (\theenumiii)}}
%%----------------------------------------------------------
\begin{document}
\title{Differential Geometry}
\author{Martín Prado - 201922940 }
\date{\today}

\maketitle
\section*{Exercise 1.}
Build an atlas with 2 charts over the n-dimensional sphere $\S^n := \{x \in \R^{n+1} : \|x\| = 1\}$.

\vspace{5mm}

\textbf{Solution.} Let $\epsilon_{i}$ be the $i$-th unitary vector from the standard basis, and let $R := \epsilon_{n+1}, -R = -\epsilon_{n+1}$ be the "north" and "south" poles of the sphere.

\begin{wrapfigure}{r}{0.35 \textwidth}
\centering
\resizebox{0.35\textwidth}{!}{%
\begin{circuitikz}
\tikzstyle{every node}=[font=\small]
\draw [](2.5,7.5) to[short] (12.5,7.5);
\draw [](7.5,12.5) to[short] (7.5,2.5);
\draw  (7.5,7.5) circle (2.5cm);
\draw [](7.25,10) to[short, -o] (7.5,10);
\draw[] (3.75,7.5) to[short] (3.5,7.5);
\draw (5.25,8.5) to[short, -*] (5.25,8.5);
\draw (8.25,7.5) to[short, -*] (8.25,7.5);
\node [font=\small] at (9.25,5) {p};
\node [font=\small] at (8.5,7.75) {p'};
\node [font=\small] at (8.75,10.5) {R = (0,1)};
\draw (9,5.5) to[short, -*] (9,5.5);
\draw (3.75,7.5) to[short, -*] (3.75,7.5);
\node [font=\small] at (3.5,7.75) {q'};
\draw [](7.75,5) to[short, -o] (7.5,5);
\node [font=\small] at (8.75,4.5) {-R = (0,-1)};
\node [font=\normalsize] at (12.25,7.75) {x};
\node [font=\normalsize] at (7.75,12) {y};
\draw [short] (7.5,10) .. controls (8.25,7.75) and (8.25,7.75) .. (9,5.5);
\node [font=\small] at (5,8.75) {q};
\draw [short] (7.5,10) .. controls (5.75,8.75) and (5.75,8.75) .. (3.75,7.5);
\end{circuitikz}
}%
\end{wrapfigure}

The atlas with two charts that we are going to build is the following. Let $U_+ := \S^n\backslash\{R\}$ and $U_- := \S^n\backslash\{-R\}$,
\[\mathcal{A} = \{(U_+, \hspace{2mm} \varphi_+: U_+ \mapsto \R^n),\hspace{3mm}(U_- , \hspace{2mm} \varphi_-: U_- \mapsto \R^n)\}\]

Where $\varphi_+, \varphi_-$ are the stereographic projections of the sphere with respect to the poles $R, -R$:
\[ \begin{array}{l}
\forall x = (x_1,\cdots , x_{n+1}) \in \S^{n}:\\
 \varphi_+ (x) := \left(\tfrac{x_1}{1-x_{n+1}}, \cdots, \tfrac{x_n}{1-x_{n+1}}\right)\\
\varphi_- (x) := \left(\tfrac{x_1}{1+x_{n+1}}, \cdots, \tfrac{x_n}{1+x_{n+1}}\right)
\end{array}\]
Note that $\forall x \in \S^{n}$
\[\varphi_-(x) = \left(\varphi_+ \circ \begin{bmatrix}
    \epsilon_1 & \cdots & \epsilon_n & -\epsilon_{n+1}
\end{bmatrix}\right)(x)\]
So we only need to prove that $\varphi_+$ is a homeomorphism.

\begin{enumerate}
    \item \textbf{Bijective:} Let $y = (y_1,\cdots,y_n) \in \R^n$, and $\|\cdot\| := \|\cdot\|_2$. We define the inverse function $\varphi_+^{-1}: R^n \mapsto U_+$ as it follows:
    \[\varphi_+^{-1}(y) = \left(\frac{2y_1}{\|y\|^2 + 1}, \cdots,\frac{2y_n}{\|y\|^2 + 1}, \frac{\|y\|^2 -1}{\|y\|^2 + 1}\right)\]
    Note that
    \[ \everymath{\displaystyle} \begin{array}{ll}
        \|\varphi_+^{-1}(y)\|^2 & = \frac{1}{(\|y\|^2 + 1)^2}\left[(\|y\|^2-1)^2 + \sum_{i = 1}^n 4y_i^2\right] \\
        & = \frac{1}{\|y\|^4+2\|y\|^2+1} \left[\|y\|^4 -2\|y\|^2 + 1 + 4\|y\|^2\right] \\
        & = \frac{1}{\|y\|^4+2\|y\|^2+1} (\|y\|^4+2\|y\|^2+1) \\
        & = 1
    \end{array}\]
\end{enumerate}


\end{document}
