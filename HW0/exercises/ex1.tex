%%--WARNINGS-----------------------------------------------------
% chktex-file 9  - Half closed intervals
% chktex-file 17 - Half closed intervals
% chktex-file 36 - Space in front of parenthesis?

\section{Exercise 1}
Consider a sequence of i.i.d.~random variables ${(X_i)}_{i\in\N}$ with $\E X_i = 0$ and $\Var X_i = 1$ for every $i\in \N$.

\begin{enumerate}
    \item Show with th Law of Large Numbers that,
    \[ \lim_{n\to\infty} \|X_1,\ldots,X_n\|_2 - \sqrt{n} \to 0 \]
    \begin{enumerate}[label=(\alph*)]
        \item in $\mathbb{P}$,
        \item a.e.,
        \item in distribution,
        \item Show that if $X_i \in L^p$ for some $p>1$, then it converges in $L^q$ for every $q \in [1\leq p)$.
    \end{enumerate} 
    \item Infer from the previous results that for
    \[ \text{Law}(X_1 , \ldots, X_n) \approx \text{UNI}(\sqrt{n} \S^{n-1}) \]
\end{enumerate}

\subsection*{Solution Part 1}

\begin{theorem}[Laws of Large Numbers]\label{loln}
Let $(X_i)_{i\in\N}$ be a sequence of i.i.d. random variables such that $\E X_i = \mu$ for every $i\in\N$, and let $\overline{X_n} = \frac{1}{n} \sum_{i = 1}^n X_i$. Then,
\begin{equation}\label{wloln}\tag{Weak Law of Large Numbers}
    \lim_{n\to\infty} \P\{\|\overline{X_n} - \mu\| > \varepsilon\} = 0,\;\forall \varepsilon > 0.
\end{equation}
\begin{equation}\label{sloln}\tag{Strong Law of Large Numbers}
    \P\{\lim_{n\to\infty} \overline{X_n} \neq \mu\} = 0.
\end{equation}
\end{theorem}

\begin{definition}[Convergence in probability]
Let $(X_n)_{n\in \N}$ be a sequence of random variables. We say that $X_n$ converges to X in probability i.e. $X_n \overset{p}{\to} X$ when
\[ \lim_{n\to\infty} \P\{ |X_n-X| > \varepsilon \} = 0,\hspace*{1em} \forall \varepsilon > 0. \]
\end{definition}

\begin{definition}[Convergence almost everywhere]
    Let $(X_n)_{n\in \N}$ be a sequence of random variables. We say that $X_n$ converges to X almost everywhere (or) i.e. $X_n \overset{a.e.}{\to} X$ when
    \[  \P\{ \lim_{n\to\infty} X_n  \neq X \} = 0\]
\end{definition}

\begin{remark}

\end{remark}

According to theorem~\ref{loln} and the previous definitions, since $X_i^2$
\[ \everymath{\displaystyle}
\begin{array}{cc}
    \text{(a)} & \frac{1}{n}\|X_1, \ldots, X_n\|_2^2 = \frac{1}{n} \sum_{i = 1}^n X_i^2 \to \E X^2 = \Var X - \E X = 1\\
    \text{(b)} & \frac{1}{n}\|X_1, \ldots, X_n\|_2^2 = \frac{1}{n} \sum_{i = 1}^n X_i^2 \to \E X^2 = \Var X - \E X = 1
\end{array} \]