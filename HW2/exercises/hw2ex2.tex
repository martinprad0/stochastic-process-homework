% chktex-file 9
% chktex-file 17
\section{Exercise 2}

\begin{enumerate}
    \item \textbf{Formulate} the criterion that characterizes the existence of an extreme distribution (in terms of $\ol{F}$).
    \item \textbf{Determine and justify} by this criterion if the following distribution has an extreme distribution. The cumulative distribution function is given by
    \[  F(n) := 1-\frac{C}{{(n+1)}^{\ln (n+1)}} ,\hspace*{1em} n\in\N.\]
    \item In the case that it has a limit distribution, \textbf{argue} about which should be the limit distribution.
\end{enumerate}

\subsection*{Solution Part 1}

The criterion states that for every distribution $F$ that satisfies the following condition:

\[ \lim_{x\to x_F-} \frac{\ol{F}(x)}{\ol{F}(x-)} = 1, \]

then, there exist sequences $d_n$ and $c_n$ such that, for the extreme $M_n = X_{n:n}$,
\[ \frac{M_n - d_n}{c_n} \overset{d}{\to} H, \]
for a non-degenerate distribution $H$. Furthermore, for the discrete case, the condition would be that
\[ \lim_{n\to \infty} \frac{\ol{F}(n)}{\ol{F}(n-1)} = 1, \]
which is equivalent to
\[ \lim_{n\to\infty} \frac{f(n)}{\ol{F}(n)} = 0. \]

\subsection*{Solution Part 2}

\[ \ol{F}(n) = \frac{C}{{(n+1)}^{\ln(n+1)}}. \]

\[ \everymath{\displaystyle}
\arraycolsep=1.8pt\def\arraystretch{2.5}
\begin{array}{rl}
    \ln \left( \lim_{n\to\infty} \frac{\ol{F}(n)}{\ol{F}(n-1)}  \right) & = \lim_{n\to\infty} \left( \ln \ol{F}(n) - \ln \ol{F}(n-1)\right)\\
    & = \lim_{x\to\infty} \ln^2(x+1) - \ln^{2}(x)\\
    & = \lim_{x\to\infty} (\ln(x+1) + \ln(x)) \cdot (\ln(x+1) - \ln(x))\\
    & = \lim_{x\to\infty} \frac{\ln\frac{x+1}{x}}{\ln^{-1}(x^2 + x)}\\ \text{(L'Hôpital)}
    & = \lim_{x\to\infty} \frac{-{(x^2 + x)}^{-1}}{-(2x+1){(x^2 + x)}^{-1} \ln^{-2}(x^2+x)}\\
    & = \lim_{x\to\infty} \frac{\ln^{2}(x^2+x)}{2x+1}\\ \text{(L'Hôpital)}
    & = \lim_{x\to\infty} \frac{(2x+1){(x^2+x)}^{-1}}{2}\\
    & = \lim_{x\to\infty} \frac{2x+1}{2(x^2+x)} = 0.
\end{array} \]

This implies that

\[ \lim_{n\to\infty} \frac{\ol{F}(n)}{\ol{F}(n-1)} = \exp\left[ \ln \left( \lim_{n\to\infty} \frac{\ol{F}(n)}{\ol{F}(n-1)}  \right) \right] = e^0 = 1.   \]


\subsection*{Solution Part 3}

It's clear that $F$ is not in the domain of attraction of the Weibull since its extreme isn't bounded. In order to see it cannot be either Fréchet, let's take a look to the continuous case,
\[ \ol{G}(x) = \frac{C}{{x}^{\ln(x)}} \cdot \1_{[1,\infty)}(x) = e^{\ln^2(x)} \1_{[1,\infty)}(x) \]

\[ \everymath{\displaystyle}
\arraycolsep=1.8pt\def\arraystretch{2.5}
\begin{array}{rl}
    \ln\left( \lim_{x\to\infty} \frac{\ol{G}(\lambda \cdot x)}{\ol{G}(x)} \right)
    & = \lim_{x\to\infty} \ln^2(\lambda x) - \ln^2(x)\\
    & = \lim_{x\to\infty} (\ln(\lambda x) + \ln(x)) \cdot (\ln(\lambda x) - \ln(x))\\
    & = \lim_{x\to\infty} \ln(\lambda x^2) \ln(\lambda)\\
    & = \infty.
\end{array} \]

Thus, it's clear that $G \in DAM(\Lambda)$ since it's not regular. Finally, note that since $F(n) = G(\ceil{x})$ for $x \in (n-1,n]$, it follows that $F$ and $G$ have equivalent tails, and thus, $F$ inherits the same domain of attraction that $G$ has.






