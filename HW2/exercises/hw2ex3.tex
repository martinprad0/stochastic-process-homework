% chktex-file 9
% chktex-file 17
\section{Exercise 3}

Let ${(X_i)}_{i\in\N}$ be a sequence of i.i.d.~random variables with $X_1 \sim \mu$,
\[ M_n := \max\left\{ X_1,\ldots, X_n \right\},\hspace*{1em} \mbox{and} \hspace*{1em} N_n := \min\left\{ X_1,\ldots,X_n \right\}. \]
\begin{enumerate}
    \item For $\mu = \mbox{Gamma}(\alpha,\beta),\; \alpha,\beta > 0$, \textbf{determine and justify} the extreme distribution of $(M_n)$ and $(N_n)$.
    \item For $\mu = \mbox{Beta}(\alpha,\beta),\; \alpha,\beta > 0$, that is $f(x) = C_{\alpha,\beta}x^{\alpha-1}{(1-x)}^{\beta -1},\; \alpha,\beta > 0$, \textbf{determine and justify} the extreme distribution of $(M_n)$ and $(N_n)$.
    \item For $\mu$ such that for $\alpha > 0$,
    \[ F(x) = \begin{cases}
    0, & \mbox{for }x< 1,\\
    \frac{\ln(x)}{x^\alpha}, & \mbox{for }x\geq 1.
    \end{cases} \]
    \textbf{determine and justify} the extreme distribution of $(M_n)$ and $(N_n)$
\end{enumerate}

\subsection*{Solution Part 1}

The gamma distribution with parameters $\alpha, \beta > 0$ has the following density function:
\[ f(x) = \frac{\beta^\alpha}{\Gamma(\alpha)} x^{\alpha - 1} e^{-\beta x^2}, \; x \in [0,\infty). \] % chktex 9

It's clear that the Gamma distribution has extreme distribution convergence since it's absolutely continuous. The extreme distribution cannot be Weibull because the population maximum $x_F = \infty$. Now, we are going to prove that the tails are not regular, and thus, the extremes do not converge to Fréchet (using L'Hôpital rule):

\[ \everymath{\displaystyle}
\arraycolsep=1.8pt\def\arraystretch{2.5}
\begin{array}{rl}
    \lim_{x\to\infty} \frac{\ol{F}(\lambda \cdot x)}{\ol{F}(x)} & = \lim_{x\to\infty} \frac{\lambda \cdot f(\lambda x)}{f(x)}\\
    & = \lim_{x\to\infty} \frac{\lambda^{\alpha} x^{\alpha -1 } e^{-\beta \lambda^2 x^2}}{x^{\alpha - 1} e^{-\beta x^2}}\\
    & = \lim_{x\to\infty} \lambda^{\alpha} \exp\left( -\beta x^2 (1-\lambda^2)  \right)\\
    & = 0,\; \forall \lambda > 0.
\end{array} \]

Therefore, the only option left for $M_n$ is to be in the Gumbell's domain of attraction, that is $F \in DAM(\Lambda)$.

Now, for $N_n$, we are going to use the von Mises condition to prove that $F$ is in the domain of attraction of the Weibull distribution. In the first place, note that the population minimum $x_F$ equals $0$ and we are approaching it from the right. Thus, after reflecting the whole distribution over the $y$ axis, we obtain

\[ \everymath{\displaystyle}
\arraycolsep=1.8pt\def\arraystretch{2}
\begin{array}{rl}
    f'(x) & = (\alpha-1)x^{\alpha-2} e^{-\beta x^2} - x^{\alpha-1} \cdot 2\beta x e^{-\beta x^2}\\
    & = x^{\alpha-2} e^{-\beta x^2}((\alpha-1)-2\beta x^2)
\end{array} \]

\[ \everymath{\displaystyle}
\arraycolsep=1.8pt\def\arraystretch{2.5}
\begin{array}{rl}
    \lim_{x\to 0^-} \frac{x f(x)}{F(x)}& = \lim_{x\to 0^-} \frac{xf'(x) + f(x)}{f(x)}\\
    & = \lim_{x\to 0^-} \frac{x^{\alpha-1} e^{-\beta x^2}((\alpha-1)-2\beta x^2)}{x^{\alpha-1} e^{-\beta x^2}} + \frac{f(x)}{f(x)}\\
    & = \lim_{x\to 0^-} (\alpha - 1) - 2\beta x^2 +1 = \alpha.
\end{array} \]

Thus, $\ol{F}(-\cdot) \in DAM(\Psi_\alpha)$.

\subsection*{Solution Part 2}

Both $M_n$ and $N_n$ will be in the domain of attraction of Weibull. For $M_n$ the von Mises condition is satisfied on the following limit:

\[ \everymath{\displaystyle}
\arraycolsep=1.8pt\def\arraystretch{2.5}
\begin{array}{rl}
    f'(x) & = (\alpha-1) x^{\alpha-2}{(1-x)}^{\beta-1} - (\beta -1) x^{\alpha-1}{(1-x)}^{\beta-2}\\
    & = x^{\alpha-2}{(1-x)}^{\beta-2}((\alpha-1)(1-x)-(\beta-1)x)
\end{array}\]

\[ \everymath{\displaystyle}
\arraycolsep=1.8pt\def\arraystretch{2.5}
\begin{array}{rl}
    \lim_{x\to 1^+} \frac{(1-x)f(x)}{\ol{F}(x)} & = \lim_{x\to 1^+} \frac{(1-x)f'(x)- f(x)}{-f(x)}\\
    & = \lim_{x\to 1^+} \frac{(1-x) \cdot x^{\alpha-2}{(1-x)}^{\beta-2}((\alpha-1)(1-x)-(\beta-1)x)}{-x^{\alpha - 1}{(1-x)}^{\beta-1}} + 1\\
    & = \lim_{x\to 1^+} -(\alpha-1)\underbrace{\frac{1-x}{x}}_{\to 0}+(\beta-1) + 1\\
    & = \beta - 1 + 1 = \beta.
\end{array} \]
Now, for $N_n$, after reflecting the whole distribution, we obtain:
\[ \everymath{\displaystyle}
\arraycolsep=1.8pt\def\arraystretch{2.5}
\begin{array}{rl}
    \lim_{x\to 0^-} \frac{x f(x)}{F(x)}& = \lim_{x\to 0^-} \frac{xf'(x) + f(x)}{f(x)}\\
    & = \lim_{x\to 0^-} \frac{x \cdot x^{\alpha-2}{(1-x)}^{\beta-2}((\alpha-1)(1-x)-(\beta-1)x)}{x^{\alpha - 1}{(1-x)}^{\beta-1}} + 1\\
    & = \lim_{x\to 0^-} (\alpha - 1) - (\beta -1)\underbrace{\frac{x}{1-x}}_{\to 0} + 1\\
    & = \alpha - 1 + 1 = \alpha.
\end{array} \]

For $M_n$ is Weibull with parameter $\beta$ and for $N_n$ is Weibull with parameter $\alpha$.
\[ F \in DAM(\Psi_\beta),\hspace*{1em} \ol{F}(-\cdot) \in DAM(\Psi_\alpha). \]

\subsection*{Solution Part 3}

The provided distribution function cannot be a distribution function since is not monotonically increasing, as in fact, it goes to 0 when $x \to \infty$. I'm going to assume that it's the density function, but even then, we would encounter some problems. After some tinkering, I believe the correct density and distribution functions should be
\[ f(x) = \alpha^2 \frac{\ln(x)}{x^{\alpha+1}} \1_{[1,\infty)}(x), \]

\[ F(x) = 1- \frac{\alpha\ln(x) + 1}{x^{\alpha}}\1_{[1,\infty)}(x). \]

For $M_n$ we are using the von Mises condition to prove it is in the domain of attraction of Fréchet:

\[ \everymath{\displaystyle}
\arraycolsep=1.8pt\def\arraystretch{2.5}
\begin{array}{rl}
    \lim_{x\to\infty} \frac{xf(x)}{\ol{F}(x)}  
    & = \lim_{x\to\infty} \alpha^2\frac{\ln(x)}{x^{\alpha}} \cdot \frac{x^{\alpha}}{\alpha\ln(x) + 1}\\
    & = \lim_{x\to\infty} \frac{\alpha^2 \ln(x)}{\alpha\ln(x)+1}\\ (\text{L'Hôpital})
    & = \lim_{x\to\infty} \frac{\alpha^2 x^{-1}}{\alpha x^{-1}}\\
    & = \alpha > 0.
\end{array}\]

Thus, $F \in DAM(\Phi_\alpha)$. Now, for $N_n$ we are going to use the von Mises condition for the Weibull domain of attraction. After re-centering and reflecting over the $y$ axis we obtain,

\[ \everymath{\displaystyle}
\arraycolsep=1.8pt\def\arraystretch{2.5}
\begin{array}{rl}
    \lim_{x\to 1-} \frac{(x-1)f(x)}{F(x)} 
    & = \lim_{x\to 1-} \frac{\alpha^2 (x-1)\ln(x)}{x^{\alpha+1}} \cdot \frac{x^{\alpha}}{x^{\alpha} - \alpha \ln(x) + 1}\\
    & = \lim_{x\to 1-} \frac{1}{x} \cdot \frac{\alpha^2 \ln(x)x -\ln(x)}{x^\alpha-\alpha\ln(x) - 1}\\
    & = \frac{1}{1} \cdot \frac{0}{1-1}.
\end{array} \]

After applying L'Hôpital rule,

\[ f'(x) = \frac{\alpha^2 (1 - (a+1)\ln(x))}{x^{\alpha+2}} \]

\[ \everymath{\displaystyle}
\arraycolsep=1.8pt\def\arraystretch{2.5}
\begin{array}{rl}
    \lim_{x\to 1-} \frac{(x-1)f(x)}{F(x)} 
    & = \lim_{x\to 1-} \frac{(x-1)f'(x) + f(x)}{f(x)}\\
    & = \lim_{x\to 1-} \frac{(x-1)\alpha^2 (1 - (a+1)\ln(x))}{x^{\alpha+2}} \cdot  \frac{x^{\alpha+1}}{\alpha^2\ln(x)} + 1\\
    & = \lim_{x\to 1-} \frac{x-1}{x} \cdot \frac{1-(\alpha + 1)\ln(x)}{\ln(x)} + 1\\
    & = \lim_{x\to 1-} \frac{(x-1)}{x\ln(x)} - (\alpha + 1)\frac{(x-1)\ln(x)}{x \ln(x)} + 1\\ \text{(L'Hôpital)}
    & = \lim_{x\to 1-} \frac{1}{\ln(x) + 1} - (\alpha + 1)\frac{1+\ln(x)-x^{-1}}{\ln(x)+1} + 1\\
    & = \frac{1}{0+1} - (\alpha+1) \frac{1+0-1}{0+1} + 1\\
    & = 2.
\end{array} \]

In the end we obtained that $N_n$ is in the domain of attraction of the Weibull, that is $\ol{F}(-\cdot)\in DAM(\Psi_2)$


