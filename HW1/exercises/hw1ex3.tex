\section{Exercise 3}

For the following theorem, give two examples where they satisfy item 2 and two examples where they don't.

\begin{theorem}
    Let $F$ be a probability distribution and $\ol{F}(x) = 1-F(x)$. Then, define
    \[ x_F = F^{-1}(x) = \inf\{t > 0 \;|\; F(t) = 1\}. \]
    If these two conditions are satisfied:
    \begin{itemize}
        \item $x_F \leq \infty$,
        \item $\tau \in (0,\infty)$,
    \end{itemize}
    Then, the following items are equivalent
    \begin{enumerate}
        \item There exists a sequence ${(u_n)}_{n\in\N}$ such that
        \[ \lim_n n \cdot \ol{F}(u_n) = \tau. \]
        \item 
        \[ \lim_{x\to x_F-} \frac{\ol{F}(x)}{\ol{F}(x-)}. \]
    \end{enumerate}
    Note that if $F$ is discrete and $x_F = \infty$, then item 2.~is equivalent to
    \[ \lim_{n \to \infty} \frac{f(n)}{\ol{F}(n)} = 0.\]
\end{theorem}
\subsection*{Solution Good Examples}

\begin{enumerate}
    \item \textbf{Zeta Distribution:} For this distribution, let $s \in (1,\infty)$ and $k \in \Z^+$. Then, for $X\sim \text{Zeta}(s)$,
    \[ f(k) = \frac{1}{k^s \zeta(s)}, \]
    \[ F(k) = \P\{X \leq k\} =\dfrac{\sum_{i = 1}^k i^{-s}}{\zeta(s)}= \dfrac{\sum_{i = 1}^k i^{-s}}{\sum_{i = 1}^\infty i^{-s}}.\]
    Thus,
    \[  \ol{F}(k) = 1-\dfrac{\sum_{i = 1}^k i^{-s}}{\sum_{i = 1}^\infty i^{-s}} = \dfrac{\sum_{i = k+1}^\infty i^{-s}}{\sum_{i = 1}^\infty i^{-s}} = \dfrac{\sum_{i = k+1}^\infty i^{-s}}{\zeta(s)} \]
    Asymptotically, 
    \[ \ol{F}(k) \sim \int_{k}^{\infty} x^{-s} dx = \frac{1}{s-1}k^{-s+1}.\]
    Therefore,
    \[ \frac{f(k)}{\ol{F}(k)} \sim \frac{1}{s-1} \frac{k^{-s}}{{k}^{-s+1}} \sim k^{-1} \to 0 \]
    \item \textbf{Gauss-Kuzmin Distribution:} For this distribution,
    \[ f(k) = -\log_2\left[ 1-\frac{1}{{(k+1)}^2} \right] .\]
    \[ \ol{F}(k) = \log_2\left( \frac{k+2}{k+1} \right) \]
    Thus,
    \[ \everymath{\displaystyle}
    \arraycolsep=1.8pt\def\arraystretch{2.4}
    \begin{array}{rl}
        \lim_{k\to\infty} \frac{\ol{F}(k)}{\ol{F}(k-1)} & = \lim_{k\to\infty}\frac{\log(k+2)-\log(k+1)}{\log(k+1)-\log(k)}\\
        & = \lim_{k\to\infty} \dfrac{\frac{1}{k+2} - \frac{1}{k+1}}{\frac{1}{k+1} - \frac{1}{k}}\\
        & = \lim_{k\to\infty} \frac{(k+1) - (k+2)}{(k+2)\cdot (k+1)} \cdot \frac{(k+1)\cdot (k)}{(k) - (k+1)}\\
        & = \lim_{k\to \infty} \frac{-k^2-k}{-k^2-3k-2} = 1
    \end{array}\]
\end{enumerate}

\subsection*{Solution Bad Examples}

\begin{enumerate}
    \item \textbf{Log Distribution:} For $0 <p < 1$, we define: 
    \[ F(k) = 1+\frac{\beta_k}{\ln(1-p)},\hspace*{1em} \ol{F}(k) = \frac{\beta_k}{-\ln(1-p)}, \]
    \[ f(k)= \frac{-p}{\ln(1-p)(1-p)}, \]
    where 
    \[ \beta_k = \beta(p; k+1, 0) = \int_{0}^p t^{k}{(1-t)}^{-1} dt ,\]
    \[ \frac{f(k)}{\ol{F}(k)} = \frac{p}{(1-p)} \cdot \frac{1}{\beta_k}.\]
    The series expansion of $\beta_k$ is the following
    \[ \beta_k = \frac{p^{k+1}}{k+1} + \frac{p^{k+2}}{k+2} + \frac{p^{k+3}}{k+3} + \cdots = \sum_{i = k+1}^\infty \frac{p^{i}}{i}. \]
    This series converges for every $k\in\N$ since $p < 1$ (ratio test). Also, the sequence $\beta_k$ is decreasing because less positive terms are being summed when $k$ is increased, and thus, $\lim_k \beta_k = 0$. Therefore,
    \[ \lim_k \frac{p}{(1-p)} \cdot \frac{1}{\beta_k} = \infty. \]

    
\end{enumerate}

\subsection*{Failed Attempts}

\begin{enumerate}
    \item \textbf{Zipf-Mandelbrot Distribution:} Note that what played in favor for the last example was the heaviness of the tails. In this next case, something similar will happen. For $N \in \Z^+, q \geq 0, s > 0$, a random variable $X_{N,q,s}$ with the Zipf-Mandelbrot distribution has the following properties,
    \[ f(k) = f_{N,q,s}(k) = \frac{1}{{(k+q)}^s}\cdot \frac{1}{H_{N,q,s}}, \]
    where
    \[ H_{N,q,s} = \sum_{i = 1}^N \frac{1}{{(i+q)}^s}. \]
    As a matter of fact, when $q = 0$ and $N\to\infty$, $X_{N,0,s}$ converges in distribution to a zeta distribution. Now,
    \[ F(k) = \frac{H_{k,q,s}}{H_{N,q,s}} \]
    and thus,title
    \[ \ol{F}(k) = \frac{\sum_{i = k+1}^N{(i+q)}^{-s}}{H_{N,q,s}}.\]
    Asymptotically,
    \[ \ol{F}(k) \sim \int_{k}^N {(x+q)}^{-s} dx = \frac{1}{s-1}\left( k^{1-s} - N^{1-s} \right). \]
    Hmmmm, it seems that it has finite support.
    \item \textbf{Beta Negative Binomial:} For $\alpha, \beta > 0 \in \R$ and $r\in\N$, define
    \[ f(k) = \binom{r+k-1}{k} \frac{B(\alpha + r, \beta+k)}{B(\alpha,\beta)}, \]
    where
    \[ B(x,y) = \frac{\Gamma(x)\Gamma(y)}{\Gamma(x+y)}. \]
    Then, by Stirling's approximation we have that
    \[ f(k) \sim \frac{\Gamma(\alpha + r )}{\Gamma(r)B(\alpha,\beta)} \frac{k^{r-1}}{{(\beta+k)}^{r+\alpha}}\]
    Thus, we can approximate the tail of the distribution with the following integral:
    \[ C = \frac{\Gamma(\alpha + r )}{\Gamma(r)B(\alpha,\beta)}, \]
    \[ \everymath{\displaystyle}
    \arraycolsep=1.8pt\def\arraystretch{2.5}
    \begin{array}{rl}
        \ol{F}(n) & \sim C\int_{n}^{\infty} \frac{x^{r-1}}{{(x-\beta)}^{r+\alpha}} dx\\
        & = C \frac{x^r \cdot {}_2F_1(1,1-a;r+1; \tfrac{x}{b}) }{br{(x-\beta)}^{r+a-1}}
    \end{array} \]
    Honestly, I tried many things to make a direct proof of the polynomial behavior of the tails. However, I couldn't find any literature on this topic. The argument I'm trying to make is that while $f(n) \sim K n^{1-\alpha}$, the tail should behave like a higher degree polynomial, and thus,
    \[ \lim_n \frac{f(n)}{\ol{F}(n)} = 0.\]
    This argument could make sense in the case where $r = \beta = \alpha = 1$, because 
    \[ f(k) \sim \binom{k}{k} \frac{B(2,k+1)}{B(1,1)} = \frac{1}{(k+1)(k+2)} = \text{Cau}(k+1)\]
    \[ \ol{F(k)} \sim \]
\end{enumerate}