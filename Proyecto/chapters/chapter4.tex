% chktex-file 3 chktex-file 40 chktex-file 37 chktex-file 9
\section{Martingalas}

\begin{definition}
    A una sucesión $Z_1,Z_2,\ldots$ de variables aleatorias se le considera un martingalas con respecto a una filtración $\mathscr{F} = \{F_j : j \in \N\}$ si, para cada $i\in\N$,
    \[ \E [Z_{m+1} \;|\; F_m] = Z_m. \]
\end{definition}

\begin{definition}
    A una sucesión $Z_1,\ldots, X_m$ de martingalas se le considera acotada si existe $c_i > 0$, tal que para cualquier $i < m$,
    \[ |Z_{i+1} - Z_i| \leq c_i, \]
    con probabilidad 1.
\end{definition}

\begin{theorem}[Desigualdad de Azuma-Hoeffding] Sea $Z_1,\ldots, Z_m$ un martingalas acotado por una sucesión $\{c_i\}_{i\in\N}>0$, entonces
    \[ \P\{Z_m - Z_1 \geq \varepsilon\} \leq \exp\left( \frac{-\varepsilon^2}{2\sum_{i = 1}^m c_i^2} \right). \]    
\end{theorem}

\begin{proof}[]
    \vspace*{-2em}
\end{proof}

A partir de ahora, suponga que $k \geq 0$ para que el proceso pueda continuar de forma indefinida

\begin{definition} Llame $M_m$ al promedio muestral del proceso de Polya y $Z_m$ a la proporción de bolas rojas que quedan en la urna después de $m$ turnos.
    \[ M_m = \frac{1}{m} \sum_{i = 1}^m  X_i = \frac{1}{m} Y_m,\]
    \[ Z_m = \frac{r+kY_m}{n+km} = \frac{r+ km M_m}{n+km}. \]
\end{definition}

Note que en caso de existir un límite en probabilidad para $M_m$, $Z_m$ tendría el mismo límite.

Ahora, con estas herramientas, el objetivo es probar que la sucesión de variables $Z_m$ converge c.s.~a una variable aleatoria $Z_\infty$ con distribución beta. Para esto vamos a probar que son un martingalas y luego usar el teorema de convergencia de Doob para probar que existe la variable aleatoria límite.

\begin{theorem}
    Las variables $Z_1,Z_2,\ldots$ definen un martingalas con respecto a la filtración $\mathscr{F}_m = \sigma(X_1,\ldots, X_m)$.
\end{theorem}
\begin{proof} Para la última linea usamos el teorema anterior
    \[ \everymath{\displaystyle}
    \arraycolsep=1.8pt\def\arraystretch{3}
    \begin{array}{rcl}
        \E [Z_{m+1} \;|\; F_m] & = & \E \left[ \frac{r+k Y_{m+1}}{n+k(m+1)} \;\Big|\; F_m \right]\\
        & = & \frac{ \E\left[r+k (Y_{m}+ X_{m+1})\;\Big|\; F_m \right]}{n+km+k}\\
        & = & \frac{ r+k\E\left[Y_{m}+ X_{m+1}\;\Big|\; X_1,\ldots,X_m \right]}{n+km+k}\\
        & = & \frac{ r+k\left[Y_{m}+\E [X_{m+1}\;|\; X_1,\ldots,X_m] \right]}{n+km+k}\\
        & = & \frac{ r+k\left[Y_{m}+\P \{X_{m+1} = 1\;|\; X_1,\ldots,X_m\} \right]}{n+km+k}\\
        & = & \frac{ r+k\left[Y_{m}+ \frac{r+ky}{n+km} \right]}{n+km+k}\\
    \end{array} \]
    Después de simplificar obtenemos
    \[  = \frac{r+kY_m}{n+km} = Z_m. \]
\end{proof}

\begin{theorem}[Teorema de Doob para convergencia de martingalas]
    Sea $Z_1, Z_2,\ldots,$ un martingalas que satisface
    \[ \sup_n \E[|Z_m|] < \infty. \]
    Entonces, existe $Z_\infty$ tal que $Z_m \to Z_\infty$ casi siempre y $Z_\infty$ tiene esperanza finita.
\end{theorem}
\begin{proof}[]
\vspace*{-1em}
\end{proof}

\begin{theorem}
    $M_m$ converge en distribución a una variable aleatoria $P$ con distribución beta, $P \sim \beta(r/I, g/I)$.
\end{theorem}
\begin{proof}
    Usamos teorema~\ref{Y-distribution}
    \[ \P\{M_m \leq x\} = \P\{Y_m \leq mx\} = \sum_{y = 0}^{\floor{mx}} \binom{m}{y} \frac{\beta\left( \frac{r}{I} + y, \frac{g}{I} + m - y \right)}{\beta\left( \frac{r}{I}, \frac{g}{I} \right)}   \]
    Luego, usamos la fórmula explícita de la función beta, para abreviar usamos $a = r/I$, $b = g/I$
    \[ \everymath{\displaystyle}
    \arraycolsep=1.8pt\def\arraystretch{3}
    \begin{array}{rcl}
        \P\{M_m \leq x\} & = & \frac{1}{\beta\left( a, n \right)} \sum_{y = 0}^{\floor{mx}} \binom{m}{y} \beta\left( a + y, b + m - y \right) \\
        & = & \frac{1}{\beta\left( a, b \right)} \sum_{y = 0}^{\floor{mx}} \binom{m}{y} \int_0^1 p^{a + y-1} (1-p)^{b + m - y-1} dp\\
        & = & \frac{1}{\beta\left( a, b \right)} \int_0^1 \left( \sum_{y = 0}^{\floor{mx}} \binom{m}{y} p^y (1-p)^{m-y} \right) p^{a-1} (1-p)^{b-1} dp
    \end{array} \]
    Defina una variable aleatoria $B_m \sim \hbox{Bin}(m,p)$ y note que la sumatoria de la expresión anterior es equivalente a $\P\{B_m \leq mx\} = \P\{B_m/m \leq x\}$. Note que $B_m/m$ es promedio muestral de $m$ Bernoulli's y por la ley de grandes números, $\P\{B_m/m \leq x\}$ converge a $\1\{p \leq x\}$. Por ende, usando el teorema de convergencia dominada, finalmente obtenemos
    \[ \P\{M_m \leq x\} \to \frac{1}{\beta\left( a, b \right)} \int_0^1  p^{a-1} (1-p)^{b-1} dp = \hbox{Beta}(a,b)\]
    Por lo tanto, $M_n$ converge en distribución a la distribución $\hbox{Beta}(a,b) = \hbox{Beta}(r/I,g/I)$.
\end{proof}

\begin{corollary}
    la sucesión $Z_m$ converge c.s. a una variable aleatoria $Z_\infty$ con distribución $\hbox{Beta}(r/I,g/I)$.
\end{corollary}
\begin{proof}
    Note que $Z_m$ y $M_m$ tienen el mismo límite en distribución. Por lo tanto, usando el teorema anterior $Z_m$ converge en distribución a la distribution $\hbox{Beta}(r/I,g/I)$. Por otra parte, $\E[|Z_m|] \leq 1$, por ende, usando el teorema de Doob, probamos que el límite $Z_\infty$ existe, y como este límite es único debe tener distribución $\hbox{Beta}(r/I,g/I)$.
\end{proof}
