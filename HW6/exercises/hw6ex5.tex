% chktex-file 3 chktex-file 9 chktex-file 17
\section{Exercise 5}

For a random variable $X\sim \hbox{Geo}(p)$ with the geometric distribution
\[ \everymath{\displaystyle}
\arraycolsep=1.8pt\def\arraystretch{2.5}
\begin{array}{rcl}
    \P\{X \geq n+m \;|\; X \geq m\} & = & \frac{ \P\{X \geq n+m \;\land\; X \geq m\}}{ \P\{X \geq m\}}\\
    & = & \frac{ \P\{X \geq n+m\} }{ \P\{X \geq m\}}\\
    & = & \frac{(1-p)^{n+m}}{(1-p)^m}\\
    & = & (1-p)^n= \P\{X \geq n\} \\
\end{array} \]

This proves that the process is memoryless. Now, we define a sequence of i.i.d.~random variables $(X_i)_{i\in\N} \sim \hbox{Geo}(p)$. Also, denote the $k$-th renovation time as $T_k = \sum_{i = 1}^{k} X_i$, which has a negative binomial distribution since it's the sum of geometric random variables.
\[ T_k \sim \hbox{BinNeg}(k,p). \]
For the trial $n$, we define the counting variable $N(n)$ as follows,
\[ N(t) = \max\{n\in\N \;:\, T_n \leq t\} \]
This variable counts the number of renovations that occur in $n$-trials. Thus, it has the binomial distribution
\[ N(t) \sim \hbox{Bin}(n,p) \]